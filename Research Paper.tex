\documentclass[]{report}
\usepackage{todonotes}
\usepackage{enumerate}
\usepackage{graphicx}
\usepackage[margin=1.5in]{geometry} 
\usepackage[utf8]{inputenc} 
\usepackage[english]{babel}  
\usepackage[backend=bibtex,style=authoryear,bibstyle=authoryear,
sorting=nyt,firstinits=true]{biblatex}
\usepackage{comment}
\usepackage{hyperref} 

\setlength\parindent{0pt}

\pagenumbering{roman}
\begin{document}
% Title Page
\title{Research in " How to increase firmware security? " }
\author{Svetoslav Stoyanov \\
	Software Engineering \\
	Fontys Venlo Hogeschool}

\maketitle

\tableofcontents
%body
\pagenumbering{arabic}
\chapter{Introduction}
\section{Research Design}
The conducted research is mainly about firmware security but the topic of embedded devices will also be mentioned because the two work together.
Embedded devices are something usual for many households and their number is expected to greatly increase in the next few years. Because of that they have lately become the usual target for security breaches. If an attacker manages to access an embedded device's firmware (source code) he can exploit that access in order to exploit other parts of the eco system.

\subsection{Purpose of the study} 
The goal of this research is to find a way to better protect firmware code from being accessed or edited by unauthorized entities.
\chapter{References}

www.mdpi.com/2076-3417/10/11/4015/htm 
\\
www.link-springer-com.fontys.idm.oclc.org/chapter/10.1007%2F978-1-4842-4300-8_7 
\\
www.s3.eurecom.fr/docs/bh13us\_zaddach.pdf



\end{document}