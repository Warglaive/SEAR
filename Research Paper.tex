\documentclass[]{report}
\usepackage{todonotes}

% Title Page
\title{Research in Firmware Security}
\author{Stoyanov}


\begin{document}
\maketitle

\begin{abstract}
	In recent years embedded devices had become very popular in many countries worldwide, they are used in households, factories, and even in infrastructure objects. Their usage is predicted to increase steadily in the next decades. Our focus is that all embedded devices run on Firmware, which is basically computer software that is meant to work with specific hardware. The Firmware is not as secure as we would like it to be, there are many ways of hacking, reverse engineering it so that one can find how exactly it operates and if there are any holes that can be exploited to benefit or harm either the users or the creators of the firmware.
\end{abstract}
This is the beginning.
\section{Introduction}
\section{Research Design}
\section{Data collection}
\section{Analysis}



\todo{Make a cake \ldots},
Make a cake . . . which renders like. The \todo command has this structure: \todo{zastrelqi indieca}. The todo text is the text that will be shown in the todonote and in the
list of todos. The optional argument options, allows the user to customize the
appearance of the inserted todonotes. For a description of all the options see
section 1.3.
\missingfigure The \missingfigure command inserts an image containing an attention sign
and the given text. The command takes only one argument \missingfigure{htexti},
a text string that could describe what the figure should consist of. An example of
its usage could be
\missingfigure{Make a sketch of the structure of a trebuchet.}

\end{document}          