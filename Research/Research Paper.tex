\documentclass[]{report}
\renewcommand\thesection{\arabic{section}}%for page numbering in arabics
\usepackage{graphicx,tabularx}%for figures and tables
\usepackage[utf8]{inputenc} %allows special characters such as ä, ö, ỳ
\usepackage[english]{babel}  %set the language to English
\usepackage[margin=1.5in]{geometry} %change page margins 
\usepackage{sectsty}%section headers
\allsectionsfont{\sffamily\large}
\subsectionfont{\sffamily\normalsize}
\linespread{1.2}% line distance
\usepackage{lipsum}% http://ctan.org/pkg/lipsum
\usepackage{caption}%use for captions on tables
%use this exact command. The style and bibliographystyle has to be authoryear (Havard). The sorting is nyt: name, year, title so that the bibliography is sorted alphabetically. firstinits=true shortens the names: Albert Einstein -> A. Einstein
\usepackage[backend=bibtex,style=authoryear,bibstyle=authoryear,sorting=nyt,firstinits=true]{biblatex}
\setlength\parindent{0pt}%include this so that your paragraphs don't indent automatically
\addbibresource{report.bib} %this attaches your bib-file, your bibliography (must be in the same folder)
\usepackage[compact]{titlesec}%include title formatting package
\usepackage{soul}

% Title Page
\title{Applied Research in Firmware security of embedded devices}
\author{Jan Polfers and Svetoslav Stoyanov}
\date{November 30th 2020 \\Module: SEAR \\Venlo, Limburg, Netherlands}


\begin{document}

\maketitle

\begin{abstract}

\begin{itemize}
	\item Complete, but very succinct summary of the paper
	\item Half a page long
	\item Short description of research(brief statements of the purpose, methods, results and conclusions)
\end{itemize}
\pagenumbering{roman}
\end{abstract}

\tableofcontents
\setcounter{page}{3}
\listoffigures %UNCOMMENT IF YOU HAVE FIGURES
%\listoftables %UNCOMMENT IF YOU HAVE TABLES
\pagebreak

\pagenumbering{arabic}	
	
\section{Introduction}

The research we are conducting is about firmware security mainly but we will touch on the topic of embedded devices also because the two work together. In general, most devices with firmware are relatively easy to reverse engineer by hackers who want to find vulnerabilities that they can exploit to attack different parts of the software eco-system. Therefore the goal of this research is to find a way to increase firmware security.

In recent years embedded devices have become very popular in many countries worldwide, they are often used in households, factories, and even in infrastructure objects. Their usage is predicted to increase steadily in the next decades. The firmware used in all embedded devices is basically computer software that is meant to work with specific hardware in order to execute one or a few very concrete functions. The Firmware is not as secure as we would like it to be, there are many ways of hacking (breaching) it so that one (the hacker) can find how exactly it operates and if there are any holes that can be exploited for benefit or to be used to harm either the users or the creators of the firmware. Even though there are many ways of hacking an embedded device, the focus of this research paper is mainly on binary reverse engineering in a firmware context. Reverse engineering is the process by which an artificial object is deconstructed to reveal its designs, architecture and mainly its source code(firmware). Therefore the point of reverse engineering a device is to reach the firmware and gain knowledge of how the device operates.

"A firmware is a program or set of instructions programmed on a hardware device(embedded device) which provides the necessary instructions for how the device communicates with the other computer hardware. For the purpose to be programmed on a hardware, the firmware is usually stored in the flash ROM of a device. While ROM is "read-only" memory, the flash ROM can be erased and rewritten because it is actually a type of flash memory. Firmware can be thought of as "semi-permanent" since it remains the same unless it is updated by a firmware updater." (Christensson, Per, 2020) Anyway an attacker can also modify the firmware and inject it back on the device replacing the old firmware with new corrupted firmware code.

The physical objects containing firmware (embedded devices) and connected to a network are described by the term "Internet Of Things".

Although IEEE IoT Initiative is proceeding to draft a white
paper (Roberto Minerva, 27 May 2015) for the formal definition of IoT, there are still no
common agreements for the definition of IoT. In this article,
we define a "Thing" on IoT that indicates a physical or
virtual object which connects to the Internet and has the
ability to communicate with human users or other objects.
Along with the growth of IoT, new security issues arise
while traditional security issues become more severe. The
main reasons are the heterogeneity and the large scale of the
objects. The impact factors can be further divided into two
categories: the diversity of the “Things” and the
communication of the “Things”. It is divided into two
categories given that each of the category encounters
different security problems.
First, the security problem for the “Things” is created by
vulnerabilities produced by careless program design; this
creates opportunities for malware or backdoors installation.
Based on the heterogeneity and the scale of the “Things” in
IoT, such security problems are more complex compared to
the security problems that we have faced now.
As for the communication medium of the “Things”, it is
expected that the networking environment for IoT will be
heterogeneous. Various communication media may face
different security challenges. Overlooking these security
problems will compromise the availability of the “Things”.
As for the contents of the communication, the heterogeneous
data structure and protocols also make content protection more complex.

For the sake of this research two methods of reverse engineering embedded device with the goal of extracting its firmware in a form of binary file will be mentioned. The first method is manual and requires several tools and manual file system inspection performed by a person, while the second method is almost entirely automated and requires only one software which does all the work and presents results.

On the other side there are several ways of protecting binaries, but again only two will be used in this research. First one is "Binary Code Packing".
"Binary Code Packing" or Abstract—Packing (or executable compression) is considered as
one of the most effective anti-reverse engineering methods in
the Microsoft Windows environment. Even though many
reversing attacks are widely conducted in the Linux-based
embedded system there is no widely used secure binary code
packing tools for Linux. (M. Kim et al, 2010)
This paper presents two secure
packing methods that use AES encryption and the UPX packer
to protect the intellectual property (IP) of software from
reverse engineering attacks on Linux-based embedded system.
We call these methods: secure UPX and AES-encryption
packing. Since the original UPX system is designed not for
software protection but for code compression, we present two
anti-debugging methods in the unpacking module of the secure
UPX to detect or abort reverse engineering attacks.
Furthermore, since embedded systems are highly resource
constrained, minimizing unpacking overhead is important.
Therefore, we analyze the performance of the two packing
methods from the perspective of: code size, execution
time, and power consumption. Our analysis results show
that the Secure UPX performs better than AES-encryption
packing in terms of the code size, execution time, and power
consumption.(M. Kim et al, 2010)
						
The other method is named "Defending Embedded Systems with SoftwareSymbiotes".
A large number of embedded devices on the internet, such asrouters and VOIP phones, are typically ripe for exploitation. Little to nodefensive technology, such as AV scanners or IDS’s, are available to pro-tect these devices. We propose a host-based defense mechanism, which wecall Symbiotic Embedded Machines (SEM), that is specifically designedto inject intrusion detection functionality into the firmware of the device.A SEM or simply the Symbiote, may be injected into deployed legacyembedded systems with no disruption to the operation of the device. ASymbiote is a code structure embedded in situ into the firmware of anembedded system. The Symbiote can tightly co-exist with arbitrary hostexecutables in a mutually defensive arrangement, sharing computationalresources with its host while simultaneously protecting the host againstexploitation and unauthorized modification. The Symbiote is stealthilyembedded in a randomized fashion within an arbitrary body of firmwareto protect itself from removal. We demonstrate the operation of a genericwhitelist-based rootkit detector Symbiote injected in situ into Cisco IOSwith negligible performance penalty and without impacting the routersfunctionality. We present the performance overhead of a Symbiote onphysical Cisco router hardware. A MIPS implementation of the Sym-biote was ported to ARM and injected into a Linux 2.4 kernel, allowingthe Symbiote to operate within Android and other mobile computingdevices. The use of Symbiotes represents a practical and effective protec-tion mechanism for a wide range of devices, especially widely deployed,unprotected, legacy embedded devices.(Cui, Ang and Stolfo, Salvatore, 2011)

//2. Methods against reverse engineering binaries
//2.1: Binary Code Packing
//2.1: Symbiotes Software - https://www.researchgate.net/publication/221427496_Defending_Embedded_Systems_with_Software_Symbiotes
//Find more methods at - https://sci-hub.se/https://ieeexplore.ieee.org/abstract/document/5479571

Research question
How does the attacker gain access to the code from the firmware, inside an embedded device?

Scope

Technical background
Firmware
Binary files (Filesystem)

\begin{itemize}
	\item Provide the reader with everything they need to know to understand what you are doing and why

	\item Length: max 3 pages
	\item Theoretical background (literature review)
	\item Why the work is important
	\item Specific research question
	\item (if applicable) Hypothesis to be tested
	\item Divide into subsections
\end{itemize}



\section{Methods}
\begin{itemize}
	\item How you performed the experiment / interview / survey or how you set up your comparative analysis
	\item Length: min. 2 pages
	\item Methodology (Research strategy, material, planning): which method you used, why, how it was carried out
	\item NO results or interpretation in this section!
\end{itemize}

We should primarily focus on experiments and research papers as material. Because these two sources give us a deep understanding of the underlying structure and the causes of the problem. Every paper you read should be logged on one of the following pages. To prove our research we maybe provide interviews.

\section{Results}

Topics:
1. Reverse engineering in general introduction 
	Half a page or a page about it in general and methods against it
	
3. Reverse engineering of binaries automated and non automated (Manual and binwalk)
	This is used as an introduction into firmware reverse engineering.Firmware extraction is a good example.

4. Methods against reverse engineering binaries
	Explain a list of methods to prevent reverse engineering.
	
5. Obfuscation methods 
	Old and reliable methods and state of the art
	
6. Deobfuscation methods
	Old and reliable methods and state of the art


\begin{itemize}
	\item Share the data you found
	\item Length: min. 2 pages
	\item Describe the results (do NOT add your interpretation/analysis)
	\item Graphs, figures and tables that show your data belong in this section. Describe the graphs and explain what the reader is seeing

\end{itemize}

	
\section{Discussion}
\begin{itemize}
	\item Interpret the results from the previous section
	\item Answer your research question/s and explain if your hypothesis was proven right or not: Refer to starting point (objective research question)

	\item Length: max. 4 pages
	\item Evaluate process (Reflection: how did it go)
	\item Further research (how your research could be improved, what else could be done)
\end{itemize}


\end{document}
	
	
\printbibliography[title=References]
	Christensson, Per. "Firmware Definition." TechTerms. (2006). Accessed Dec 9, 2020. https://techterms.com/definition/firmware.
	
	M. Kim et al., "Design and Performance Evaluation of Binary Code Packing for Protecting Embedded Software against Reverse Engineering," 2010 13th IEEE International Symposium on Object/Component/Service-Oriented Real-Time Distributed Computing, Carmona, Seville, 2010, pp. 80-86, doi: 10.1109/ISORC.2010.23.
\end{document}          
