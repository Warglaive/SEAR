\documentclass[]{report}
\renewcommand\thesection{\arabic{section}}%for page numbering in arabics
\usepackage{graphicx,tabularx}%for figures and tables
\usepackage[utf8]{inputenc} %allows special characters such as ä, ö, ỳ
\usepackage[english]{babel}  %set the language to English
\usepackage[margin=1.5in]{geometry} %change page margins 
\usepackage{sectsty}%section headers
\allsectionsfont{\sffamily\large}
\subsectionfont{\sffamily\normalsize}
\linespread{1.2}% line distance
\usepackage{lipsum}% http://ctan.org/pkg/lipsum
\usepackage{caption}%use for captions on tables
%use this exact command. The style and bibliographystyle has to be authoryear (Havard). The sorting is nyt: name, year, title so that the bibliography is sorted alphabetically. firstinits=true shortens the names: Albert Einstein -> A. Einstein
\usepackage[backend=bibtex,style=authoryear,bibstyle=authoryear,sorting=nyt,firstinits=true]{biblatex}
\setlength\parindent{0pt}%include this so that your paragraphs don't indent automatically
\addbibresource{report.bib} %this attaches your bib-file, your bibliography (must be in the same folder)
\usepackage[compact]{titlesec}%include title formatting package

% Title Page
\title{Applied Research in Firmware security of embedded devices}
\author{Jan Polfers and Svetoslav Stoyanov}
\date{November 30th 2020 \\Module: SEAR \\Venlo, Limburg, Netherlands}


\begin{document}

\maketitle

\begin{abstract}

\begin{itemize}
	\item Complete, but very succinct summary of the paper
	\item Half a page long
	\item Short description of research(brief statements of the purpose, methods, results and conclusions)
\end{itemize}
\pagenumbering{roman}
\end{abstract}

\tableofcontents
\setcounter{page}{3}
\listoffigures %UNCOMMENT IF YOU HAVE FIGURES
%\listoftables %UNCOMMENT IF YOU HAVE TABLES
\pagebreak

\pagenumbering{arabic}	
	
\section{Introduction}

The research we are conducting is about firmware security mainly but we will touch on the topic of embedded devices also because the two work together. In general, most devices with firmware are relatively easy to reverse engineer by hackers who want to find vulnerabilities that they can exploit to attack different parts of the software eco-system. Therefore the goal of this research is to find a way to increase firmware security.

In recent years embedded devices have become very popular in many countries worldwide, they are often used in households, factories, and even in infrastructure objects. Their usage is predicted to increase steadily in the next decades. The firmware used in all embedded devices is basically computer software that is meant to work with specific hardware in order to execute one or a few very concrete functions. The Firmware is not as secure as we would like it to be, there are many ways of hacking (breaching) it so that one (the hacker) can find how exactly it operates and if there are any holes that can be exploited for benefit or used to harm either the users or the creators of the firmware. Even though there are many ways of hacking an embedded device, the focus of this research paper is mainly on binary reverse engineering in a firmware context. Reverse engineering is the process by which an artificial object is deconstructed to reveal its designs, architecture, code or to extract knowledge from the object. The point of reverse engineering a device is to reach the firmware.

"A firmware is a program or set of instructions programmed on a hardware device(embedded device) which provides the necessary instructions for how the device communicates with the other computer hardware. For the purpose to be programmed on a hardware, the firmware is usually stored in the flash ROM of a device. While ROM is "read-only" memory, the flash ROM can be erased and rewritten because it is actually a type of flash memory. Firmware can be thought of as "semi-permanent" since it remains the same unless it is updated by a firmware updater." (Christensson, Per, 2020) Anyway an attacker can also modify the firmware and inject it back on the device replacing the old firmware with new corrupted firmware code.

The physical objects containing firmware and connected to a network are described by what is called "Internet Of Things".

//IOT Detailed explanation
As mentioned before, the usage of embedded devices in households, factories, infrastructure objects is increasing and is 


//Why this is important but (IOT, machines in factorys etc.) it is included. More details and make sure the reader is aware of the importance of the security. Because it can affect everything.



\begin{itemize}
	\item Provide the reader with everything they need to know to understand what you are doing and why

	\item Length: max 3 pages
	\item Theoretical background (literature review)
	\item Why the work is important
	\item Specific research question
	\item (if applicable) Hypothesis to be tested
	\item Divide into subsections
\end{itemize}



\section{Methods}
\begin{itemize}
	\item How you performed the experiment / interview / survey or how you set up your comparative analysis
	\item Length: min. 2 pages
	\item Methodology (Research strategy, material, planning): which method you used, why, how it was carried out
	\item NO results or interpretation in this section!
\end{itemize}

We should primarily focus on experiments and research papers as material. Because these two sources give us a deep understanding of the underlying structure and the causes of the problem. Every paper you read should be logged on one of the following pages. To prove our research we maybe provide interviews.

\section{Results}
Topics:
1. Reverse engineering in general introduction 
	Half a page or a page about it in general and methods against it
	
2. Differences in reverse engineering between java c++ and binarys
	Maybe own chapter or just half a page I dont know
	Why methods applied for java or c++ cant be applied for binarys
	
3. Reverse engineering of binaries automated and non automated (Manual and binwalk)
	This is used as an introduction into binary reverse engineering

4. Methods against reverse engineering binarys
	Explain a list of methods to prevent reverse engineering.
	
5. Obfuscation methods 
	Old and reliable methods and state of the art
	
6. Deobfuscation methods
	Old and reliable methods and state of the art


\begin{itemize}
	\item Share the data you found
	\item Length: min. 2 pages
	\item Describe the results (do NOT add your interpretation/analysis)
	\item Graphs, figures and tables that show your data belong in this section. Describe the graphs and explain what the reader is seeing

\end{itemize}

	
\section{Discussion}
\begin{itemize}
	\item Interpret the results from the previous section
	\item Answer your research question/s and explain if your hypothesis was proven right or not: Refer to starting point (objective research question)

	\item Length: max. 4 pages
	\item Evaluate process (Reflection: how did it go)
	\item Further research (how your research could be improved, what else could be done)
\end{itemize}


\end{document}
	
	
\printbibliography[title=References]
	Christensson, Per. "Firmware Definition." TechTerms. (2006). Accessed Dec 9, 2020. https://techterms.com/definition/firmware.
\end{document}          
